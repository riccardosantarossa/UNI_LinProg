\documentclass[12pt]{extarticle}
\usepackage[top=1.5cm, bottom=1.5cm, left=1cm, right=1cm]{geometry}
\usepackage{graphicx}
\usepackage{listings}
\usepackage{color}


\definecolor{dkgreen}{rgb}{0,0.6,0}
\definecolor{gray}{rgb}{0.5,0.5,0.5}
\definecolor{mauve}{rgb}{0.58,0,0.82}

\lstset{
  frame=none,
  language=C,
  aboveskip=3mm,
  belowskip=3mm,
  showstringspaces=false,
  columns=flexible,
  basicstyle={\normalsize\ttfamily},
  numbers=none,
  numberstyle=\tiny\color{gray},
  keywordstyle=\color{blue},
  commentstyle=\color{dkgreen},
  stringstyle=\color{mauve},
  breaklines=true,
  breakatwhitespace=true,
  tabsize=3
}

\title{Tipi di Dato}
\author{R.S}

\begin{document}
\maketitle

\section*{Tipi di dato composti}

\textbf{Record}

\begin{flushleft}

Sono una collezione di campi di tipi diversi selezionabili con il proprio nome \\    
In C e in Java si rappresentano come:
\begin{itemize}
    \item Struct in C: 
    \begin{lstlisting}
      struct studente 
      {
        char nome[20];
        int matricola; 
      };
        
        struct studente s;
        s.nome = "Mario"; // errore in C
        strcpy(s.nome, "Mario");
        s.matricola=343536;
    \end{lstlisting}
    \item Classe in Java:
    \begin{lstlisting}
      class Studente 
      {
        public String nome;
        public int matricola;
      };
      
      Studente s = new Studente();
      s.nome = "Mario";
      s.matricola = 343536;
    \end{lstlisting}
\end{itemize}

Sono memorizzabili e denotabili ma non possono essere definiti senza identificatore (solo in Ada). \\
Le struct sono esprimibili \textbf{soltanto} in Scheme $\rightarrow$ liste con funzioni di accesso ai campi e 
funzioni di test del valore del campo:

\begin{lstlisting}
  (define (book title authors) (list book title authors))
  (define (book-title b) (car (cdr b)))
  (define (book-? b) (eq? (car b) book))
\end{lstlisting}

I campi del record sono memorizzati sequenzialmente e sono allineati; \\
Si possono disallineare ma l'accesso risulta più complesso $\rightarrow$ richiede l'assembler. \\
Si possono riordinare i campi per risparmiare spazio ma bisogna preservare l'ordinamento nel tempo.

\newpage

\textbf{Record varianti e Union}

\begin{itemize}
  \item record in cui solo alcuni campi sono attivi in un determinato istante
  \item alcuni campi sono alternativi tra loro $\rightarrow$  i campi alternativi possono condividere la stessa locazione di memoria
  \item possibili varianti $\rightarrow$  alloco la dimensione massima che posso ottenere \\
        a seconda della configurazione uso la locazione di memoria per rappresentare tipologie di dato differenti
  \item rappresentazione:
  \begin{lstlisting}
    type studente = record
    nome : packed array [1..6] of char;
    matricola: integer;
    case fuoricorso : Boolean of
      true: (ultimoanno: 2020..maxint);
      false:(anno:(primo,secondo,terzo);
             inpari:Boolean;) end;
    var s: studente;
    s.fuoricorso := true;
    s.ultimoanno:= 2021;
\end{lstlisting}
  \item i tipi unione possono essere utili per risparmiare spazio o per rappresentsre in modo naturale 
        alcuni tipi come le liste o gli alberi binari (es: alberi con solo un sottoalbero) \\
        in linguaggi come Rust i tipi unione vengono etichettati e viene definito il codice 
        che gestisce ogni possibile opzione
  \item con i tipi unione si possono aggirare controlli sul tipo:
        \begin{lstlisting}
          union Data {
            int i;
            float f;
            char str[20];
          } data;
          float y;
          ...
          data.str = "abcd";
          y = data.f;
        \end{lstlisting}
\end{itemize}

\textbf{Array}
\begin{itemize}
  \item insiemi di dati omogenei contraddistinti da un indice (array di caratteri $\rightarrow$ stringhe) \\
        possono anche essere multidimensionali, con più indici (array di array)
  \item posso implementare l'operazione di selezione (A[i]) e in alcuni linguaggi 
        posso anche avere lo slicing per selezionare parti contigue all'interno di un singolo array; \\
        sono memorizzati in locazioni contigue (singola dimensione) oppure in ordine di riga/colonna (multidimensionali)
  \item forma: 
        \begin{itemize}
          \item statica  $\rightarrow$ definita a compile time e fissata alla dichiarazione dell'array
          \item dinamica  $\rightarrow$ varia durante l'esecuzione e usa il \textbf{Dope Vector}
          \item Dope Vector  $\rightarrow$ descrittore che contiene: 
          \begin{itemize}
            \item puntatore all'inizio, limite inferiore e occupuazione di ogni dimensione dell'array
            \item memorizzato nella parte fissa del record di attivazione 
          \end{itemize}
          \item accesso al vettore  $\rightarrow$ calcolo dell'indirizzo base a runtime usando le informazioni del Dope Vector
          \item memorizzazione: 
          \begin{itemize}
            \item statica $\rightarrow$ nell' RdA
            \item fissata alla dichiarazione $\rightarrow$ Dope Vector nella parte iniziale che punta il vettore $\rightarrow$ vettore in fondo
            \item forma dinamica $\rightarrow$ heap, solo il dope vector si trova nell' RdA
          \end{itemize}
          \item memorizzazione in Java:
          \begin{itemize}
            \item gli arrai sono oggetti memorizzati nella heap
            \item lo spazio viene allocato con una chiamata a new
            \begin{lstlisting}
              int[] vector = new int[DIM]; 
            \end{lstlisting}
          \end{itemize}
        \end{itemize}
  \item sicurezza degli array:
  \begin{itemize}
    \item il controllo degl indici è dinamico ed è svolto solo da alcuni linguaggi
    \item è importante per la \textbf{type safety} $\rightarrow$ potrei accedere a zone con tipo sbagliato
    \item buffer overflow $\rightarrow$ tentativo di eseguire codice arbitrario cambiando gli indirizzi di ritorno
  \end{itemize}
\end{itemize}

\medskip

\textbf{Insiemi}

\begin{itemize}
  \item implementazione $\rightarrow$ vettori di booleani che definiscono la funzione caratteristica
  \item le operazioni insiemistiche si ottengono con le operazioni booleane \textbf{bit wise}
\end{itemize}

\medskip

\textbf{Puntatori}

\begin{itemize}
  \item identificano l-value ed indirizzi di memoria
  \item tipo del puntatore $\rightarrow$ tipo dell'oggetto puntato 
  \begin{lstlisting}
     int *i; //puntatore ad intero
  \end{lstlisting}
  \item operazioni possibili: 
  \begin{itemize}
    \item creazione dell'oggetto puntato $\rightarrow$ funzioni di libreria
    \item deferenziazione dell'oggetto
    \item test di uguaglianza $\rightarrow$ uguali se puntano \textbf{esattamente} alla stessa locazione di memoria
  \end{itemize}
  \item in C i puntatori fanno riferimento allo stack con l'operatore \& (ritornano l'indirizzo)
  \begin{lstlisting}
    int i = 5;
    p = &i; //ritorno l'indirizzo della variabile i
  \end{lstlisting}
  \item in C, array e puntatori son intercambiabili $\rightarrow$ il puntatore punta all'indirizzo base dell'array
  \begin{lstlisting}
    int n;
    int *a; // puntatore a interi
    int b[10]; // array di 10 interi
    ...
    a = b; // a punta a elemento iniziale di b
    n = a[3]; // n ha il valore del terzo elemento di b
    n = *(a+3); // idem
    n = b[3]; // idem
    n = *(b+3); // idem
  \end{lstlisting}
  \item aritmetica dei puntatori:
  \begin{itemize}
    \item la somma corrisponde alla somma di locazioni  $\rightarrow$  a $+$ 3 significa 3 locazioni intere in più di a
    \item non ho alcuna garanzia di correttezza perchè posso accedere a zone arbitrarie di memoria
  \end{itemize}
  
  \newpage
  
  \item utilizzo  $\rightarrow$  strutture dati ricorsive come alberi e liste concatenate:
  \begin{lstlisting}
    struct int_list {
      int info;
      struct int_list *next;
    }
    struct char_tree {     
      char info;
      struct char_tree *left, *right;
    }
  \end{lstlisting}

  \item in Java non ci sono i puntatori $\rightarrow$ definisco direttamente il tipo ricorsivo (Java lavora per riferimento)
  \item \textbf{Dangling Reference}: riferimento ad un'area di memoria che contiene dati non compatibili
  \begin{itemize}
    \item problema $\rightarrow$ deallocazione della heap e deallocazione di RdA
    \item soluzioni $\rightarrow$ oggetti solo nella heap e \textbf{garbage collection}
  \end{itemize}
  \item \textbf{Garbage collection}: recupero di spazio di memoria heap 
  \begin{itemize}
    \item è più sicura della deallocazione esplicita
    \item pesante da implementare $\rightarrow$ algoritmi complessi che rallentano l'esecuzione
    \item In Rust:
    \begin{itemize}
      \item ownership $\rightarrow$ un dato appartiene ad una variabile $\rightarrow$ non ci sono heap condivisi
      \begin{lstlisting}
        let mascot = String::from("stringaEsempio");
        let ferris = mascot.clone();
        println!("{}", mascot) // Works
      \end{lstlisting}
      \item all'uscita di una procedura si recuperano i dati tramite l'ownership
    \end{itemize}
  \end{itemize}
\end{itemize}

\medskip

\section*{Inferenza di tipo}
\begin{itemize}
  \item dichiarazione di una variabile $\rightarrow$ tipo della variabile
  \item il compilatore associa un tipo alle altre parti di codice (espressioni, procedure..)
  \item algoritmi di \textbf{type inference}:
  \begin{itemize}
    \item Python $\rightarrow$ il tipo è associato dinamicamente (può cambiare nel tempo)
    \item Haskell $\rightarrow$ il tipo è definito dal contesto, il type checking è statico
  \end{itemize}
\end{itemize}

\smallskip

\section*{Equivalenza e compatibilità tra tipi}
\begin{itemize}
  \item Equivalenza: determina se due variabili hanno lo stesso tipo
  \begin{itemize}
    \item i tipi T ed S sono \textbf{equivalenti} se ongi variabile di tipo T è anche di tipo S
  \end{itemize}
  \item Equivalenza per \textbf{nome} $\rightarrow$ sono lo stesso identificatore di tipo
  \begin{itemize}
    \item si può richiedere nel passaggio di parametri
    \begin{lstlisting}
      //valido in C
      int[4] sequenza = {0, 1, 2, 3};
      void ordina (int[4] vett);
      ...
      ordina(sequenza)
    \end{lstlisting}
  \end{itemize}
\end{itemize}

\newpage

\begin{itemize}
  \item Equivalenza \textbf{Strutturale}:
  \begin{itemize}
    \item due tipi sono equivalenti se hanno la stessa struttura
    \item se T è definito come $T = espressione$ allora T è equivalente a espressione
    \item se costruisco due tipi con lo stesso \textbf{costruttore} allora sono equivalenti
    \item esempio:
    \begin{lstlisting}
      //le struct A e B sono strutturalmente equivalenti
      struct coppiaA {int i; float f};
      struct A {int j; struct coppiaA c};
      struct B {int j; struct coppiaB {int i; float f} c};
    \end{lstlisting} 
    \item equivalenza strutturale $\rightarrow$ basso livello $\rightarrow$ non rispetta l'astrazione
  \end{itemize}
\end{itemize}

\begin{itemize}
  \item \textbf{Compatibilità} di tipo $\rightarrow$ oggetti di tipo T possono essere usati dove ci si aspetta oggetti di tipo S
  \begin{itemize}
  \item esempio:
  \begin{lstlisting}
    //int e float sono compatibili
    int n = 5
    float r = 5.2;
    r = r + n;
  \end{lstlisting}
  \end{itemize}
  \item Due tipi \textbf{equivalenti} sono anche \textbf{compatibili} (sono intercambiabili)
\end{itemize}

\section*{Conversione di tipo (cast)}
\begin{itemize}
  \item Utilizzo di specifiche funzioni per convertire un tipo in un altro
  \begin{lstlisting}
    s = (S) t
    r = (float) n;
    n = (int) r;
  \end{lstlisting}
\end{itemize}

\section*{Polimorfismo}
\begin{itemize}
  \item Polimorfismo \textbf{overloading}:
  \begin{itemize}
    \item uno stesso simbolo può denotare cose diverse a seconda del contesto
    \item viene risolto a tempo di \textbf{compilazione} dopo l'inferenza di tipo
    \item supportato in Java $\rightarrow$ classi e sottoclassi (ereditarietà)
    \item definisco una funzione più volte $\rightarrow$ si sceglie quale usare tramite il numero e il tipo degli argomenti
  \end{itemize}
  \item Polimorfismo \textbf{parametrico}:
  \begin{itemize}
    \item si usa un parametro generico 
    \item la funzione \textbf{polimorfa} ha un unico codice ma si può applicare a oggetti di tipo diverso
    \begin{lstlisting}
      identity(x) = x; : <T> -> <T>
    \end{lstlisting}
  \end{itemize}
  \item Polimorfismo in Scheme:
  \begin{itemize}
    \item type checking dinamico $\rightarrow$ linguaggio naturalmente polimorfo
    \item esempio $\rightarrow$ la funzione MAP
    \begin{lstlisting}
      (map f list) //mappo la funzione f su tutta la lista
      (T -> S) -> (list T) -> (list S)
    \end{lstlisting}
  \end{itemize}
  \newpage
  \item Polimorfismo parametrico \textbf{esplicito}:
  \begin{itemize}
    \item in C++ si usano le funzioni template
  \begin{lstlisting}
    //dichiarazione del tipo generico T
    template <typename T>
    void swap (T& x, T& y){
      T tmp = x; 
      x=y; 
      y=tmp;
    }
  \end{lstlisting}
    \item l'istanziazione del tipo giusto è automatica
    \begin{lstlisting}
      int i,j; swap(i,j); //T diventa int a link-time
      float r,s; swap(r,s); //T diventa float a link time
      String v,w; swap(v,w); //T diventa String a link time
    \end{lstlisting}
    \item in Java si usano i tipi generici:
    \begin{lstlisting}
      //si usa per qualunque tipo di Array (char, int, ...)
      public static < E > void printArray( E[] inputArray ) {
        for(E element : inputArray) {
            System.out.printf("%s ", element);
        }
      }
    \end{lstlisting} 
  \end{itemize} 
  \item Polimorfismo di \textbf{sottotipo}:
  \begin{itemize}
    \item basato sul fatto che un sottotipo condivide tutte le proprietà del suo tipo padre
    \item può essere combinato con il polimorfismo parametrico:
    \begin{itemize}
      \item solo il sottotipo: select: D$[]$ $\rightarrow$ D (restituisce sempre di tipo D)
      \item combinando i due: $\forall$ T $ [] \rightarrow T$ (per tutti i sottotipi di T)
    \end{itemize}
    \item nei linguaggi ad oggetti creo le classi generiche:
    \begin{lstlisting}
      void quickSort( E[] inputArray)
      void quickSort( E[] inputArray, (E*E)->Bool compare) //devo passare le funzioni ausiliarie

      //Uso quindi una typeclass che contengono le funzioni necessarie
      class S
      {
        bool compare (x, y : S)
      }
      void quickSort(S[] inputArray)
    \end{lstlisting}
  \end{itemize}
\end{itemize}

\end{flushleft}
\end{document}
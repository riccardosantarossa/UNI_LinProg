\documentclass[12pt]{extarticle}
\usepackage[top=1.5cm, bottom=1.5cm, left=1cm, right=1cm]{geometry}
\usepackage{graphicx}
\usepackage{listings}
\usepackage{color}
\usepackage{multicol}
\usepackage{bussproofs}


\definecolor{dkgreen}{rgb}{0,0.6,0}
\definecolor{gray}{rgb}{0.5,0.5,0.5}
\definecolor{mauve}{rgb}{0.58,0,0.82}

\lstset{
  frame=none,
  language=C,
  aboveskip=3mm,
  belowskip=3mm,
  showstringspaces=false,
  columns=flexible,
  basicstyle={\normalsize\ttfamily},
  numbers=none,
  numberstyle=\tiny\color{gray},
  keywordstyle=\color{blue},
  commentstyle=\color{dkgreen},
  stringstyle=\color{mauve},
  breaklines=true,
  breakatwhitespace=true,
  tabsize=3
}

\title{Sistemi di assegnazione di tipo}
\author{R.S}

\begin{document}
\maketitle

\section*{Sistema di tipi}
\begin{itemize}
  \item Definizione del controllo di tipi $\rightarrow$ regole di \textbf{derivazione} del tipo:
  \begin{itemize}
    \item $x_1:A_1, x_2:A_2, \dots, x_n:A_n \vdash M:A$
    \item l'\textbf{ambiente} viene definito con G ($\Gamma$)
  \end{itemize}
\end{itemize}

\section*{Derivazioni e regole di derivazione}
\begin{itemize}
  \item Utilizzo di premesse $\rightarrow$ simile alla deduzione naturale: 
  \begin{flushleft}
    \(
        \displaystyle
        \frac{G_1 \vdash M_1 : A_1 \quad \cdots \quad G_n \vdash M_n : A_n}
        {G \vdash M : A}\)
  \end{flushleft}
  \item Regole di derivazione:
  \begin{itemize}
    \item regole senza premesse $\rightarrow$ assiomi
    \item regole date per \textbf{induzione sulla struttura} di M $\rightarrow$ ogni costrutto ha la sua regola
    \item linguaggio più ricco $\rightarrow$ approccio modulare con più regole
  \end{itemize}
  \item Derivazioni:
  \begin{itemize}
    \item struttura ad \textbf{albero} $\rightarrow$ derivo i giudizi validi
  \end{itemize}
\end{itemize}

\section*{Type Checking e Inference}
\begin{itemize}
  \item Diversi problemi di un sistema di tipo: controllo, inferenza e analisi semantica
  \item Type \textbf{Checking}:
  \begin{itemize}
    \item dati un termine M, un tipo A e un ambiente $\Gamma$ $\rightarrow$ determinare se $\Gamma \vdash M:A$
  \end{itemize}
  \item Type \textbf{Inference}:
  \begin{itemize}
    \item dato un termine M ed un ambiente $\Gamma$ $\rightarrow$ trovare un tipo A tale che $\Gamma \vdash M:A$ sia valido
  \end{itemize}
  \item Analisi semantica:
  \begin{itemize}
    \item risolve problemi di type inference
    \item nel codice si definisce il tipo della variabile $\rightarrow$ necessario inferire sulle sotto-espressioni
  \end{itemize}
\end{itemize}

\newpage

\section*{Sistemi di tipo al primo ordine}
\begin{itemize}
  \item Primo ordine $\rightarrow$ non ci sono tipi polimorfi
  \item Sistema definito $\rightarrow$ F1 (lambda calcolo tipato)
  \begin{itemize}
    \item linguaggio funzionale minimale con costanti, lambda-astrazioni e applicazione di funzione
    \item costante $\rightarrow$ C
    \item astrazioni $\rightarrow$ $\lambda x:A.$ $M$
    \item applicazione di funzione $\rightarrow$ M N
  \end{itemize}
  \item Regole di derivazione dei tipi:
  \begin{itemize}
    \item Variabile: $G, x:A, G' \vdash x:A$
    \item Lambda Astrazione: 
    \begin{flushleft}
    \hspace*{7em}
    \(
        \displaystyle
        \frac
        {G, x:A \vdash M:B}
        {G \vdash (\lambda x:A. M) : A \rightarrow B }
    \)
  \end{flushleft}
  \item Applicazione di funzione:
  \begin{flushleft}
   \hspace*{7em}
    \(
        \displaystyle
        \frac
        {G \vdash M : A \rightarrow B \qquad G \vdash N:A}
        {G \vdash (\lambda x:A. M) : A \rightarrow B }
    \)
  \end{flushleft}
  \end{itemize}
\end{itemize}

\section*{Costruzione della derivazione}
\begin{itemize}
  \item Costruzione top-down: seleziono la regola da applicare $\rightarrow$ riduco il problema ai sottotermini
  \item Ad ogni passaggio:
  \begin{itemize}
    \item selezione della regola $\rightarrow$ osservo il costrutto principale
    \item istanziare lo schema di regole $\rightarrow$ meccanismo simile al pattern matching
  \end{itemize}
\end{itemize}
\end{document}
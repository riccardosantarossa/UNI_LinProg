\documentclass[12pt]{extarticle}
\usepackage[top=1.5cm, bottom=1.5cm, left=1cm, right=1cm]{geometry}
\usepackage{graphicx}
\usepackage{listings}
\usepackage{color}
\usepackage{multicol}


\definecolor{dkgreen}{rgb}{0,0.6,0}
\definecolor{gray}{rgb}{0.5,0.5,0.5}
\definecolor{mauve}{rgb}{0.58,0,0.82}

\lstset{
  frame=none,
  language=C,
  aboveskip=3mm,
  belowskip=3mm,
  showstringspaces=false,
  columns=flexible,
  basicstyle={\normalsize\ttfamily},
  numbers=none,
  numberstyle=\tiny\color{gray},
  keywordstyle=\color{blue},
  commentstyle=\color{dkgreen},
  stringstyle=\color{mauve},
  breaklines=true,
  breakatwhitespace=true,
  tabsize=3
}

\title{Programmazione Concorrente}
\author{R.S}

\begin{document}
\maketitle

\section*{Tipologie di concorrenza}

\begin{itemize}
    \item Concorrenza \textbf{fisica}
    \begin{itemize}
        \item esecuzione simultanea di istruzioni
        \item comprende diversi gradi di granularità:
        \begin{itemize}
            \item parallelismo a livello di istruzioni
            \item parllalismo a livello di processo $\rightarrow$ multicore
            \item computazione vettoriale $\rightarrow$ GPU
            \item reti di calcolatori
        \end{itemize}
    \end{itemize}
    \item Concorrenza \textbf{logica}
    \begin{itemize}
        \item programmazione parallela:
        \begin{itemize}
            \item parallelizzazione dell'esecuzione di un singolo problema 
            \item algoritmi specifici per il parallelismo
        \end{itemize}
        \item programmazione multithreaded:
        \begin{itemize}
            \item più processi attivi \textbf{contemporaneamente} sulla stessa macchina fisica
            \item memoria condivisa oppure con scambio di messaggi 
        \end{itemize}
        \item programmazione distribuita
        \begin{itemize}
            \item programmi \textbf{concorrenti} eseguiti su macchine separate
            \item memoria indipendente per ogni nodo della rete
        \end{itemize}
    \end{itemize}
    \item Cloud Computing
    \begin{itemize}
        \item insieme di tecnologie che permettono lo sviluppo di applicazioni \textbf{distribuite} sul web
        \item accesso on demand con risorse distribuite configurabili a richiesta
    \end{itemize}
\end{itemize}

\newpage

\section*{Metodi e aspetti della programmazione concorrente}

\begin{itemize}
    \item Parallelismo con librerie
    \begin{itemize}
        \item utilizzo di un linguaggio sequenziale
        \item threads con chiamate a funzioni di libreria $\rightarrow$ C e POSIX threads
    \end{itemize}
    \item Aspetti della programmazione concorrente
    \begin{itemize}
        \item comunicazione $\rightarrow$ i thread si scambiano informazioni
        \item sincronizzazione $\rightarrow$ regolazione della velocità relativa (scheduler)
    \end{itemize}
\end{itemize}

\section*{Thread e processi}

\begin{itemize}
    \item 
\end{itemize}

\end{document}
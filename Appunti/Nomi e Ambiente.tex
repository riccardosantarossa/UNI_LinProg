\documentclass[12pt]{extarticle}
\usepackage[top=1.5cm, bottom=1.5cm, left=1cm, right=1cm]{geometry}
\usepackage{graphicx}
\usepackage{listings}
\usepackage{color}
\usepackage{multicol}


\definecolor{dkgreen}{rgb}{0,0.6,0}
\definecolor{gray}{rgb}{0.5,0.5,0.5}
\definecolor{mauve}{rgb}{0.58,0,0.82}

\lstset{
  frame=none,
  language=C,
  aboveskip=3mm,
  belowskip=3mm,
  showstringspaces=false,
  columns=flexible,
  basicstyle={\normalsize\ttfamily},
  numbers=none,
  numberstyle=\tiny\color{gray},
  keywordstyle=\color{blue},
  commentstyle=\color{dkgreen},
  stringstyle=\color{mauve},
  breaklines=true,
  breakatwhitespace=true,
  tabsize=3
}

\title{Nomi e Ambiente}
\author{R.S}

\begin{document}
\maketitle

\section*{Nomi}

\begin{itemize}
  \item Meccanismi di \textbf{astrazione} per rappresentare sinteticamente altre cose: valori, locazioni di memoria\dots
  \item Nome $\rightarrow$ sequenza di caratteri identificativi:
  \begin{lstlisting}
    const pi = 3.14;
    int x = pi + 1;
    void f(){...};
  \end{lstlisting}
  \item Rappresentano oggetti \textbf{denotabili} $\rightarrow$ diversi nei vari linguaggi di programmazione
\end{itemize}

\section*{Binding e Ambiente}

\begin{itemize}
  \item Ambiente $\rightarrow$ insieme dei legami esistenti
  \item Binding (legame) $\rightarrow$ associazione tra nome e oggetto denotabile
  \item Creazione del binding:
  \begin{itemize}
    \item può venire creato in vari momenti $\rightarrow$ definizione, caricamento del programma, scrittura del codice \dots
    \item binding \textbf{statico} $\rightarrow$ prima dell'esecuzione della prima istruzione
    \item binding \textbf{dinamico} $\rightarrow$ durante l'esecuzione
  \end{itemize}
  \item Ambiente e store $\rightarrow$ definiscono il valore di una variabile
  \begin{itemize}
    \item ambiente $\rightarrow$ definisce la locazione di memoria che contiene la variabile
    \item store $\rightarrow$ determina il dato effettivo
    \item con i comandi modifico lo store, con le dichiarazioni modifico l'ambiente
  \end{itemize}
  \item Dichiarazione $\rightarrow$ meccanismo per creare un binding $\rightarrow$ modifica dell'ambiente
  \begin{lstlisting}
    int x = 0;
    typedef int T;
    int inc (T x) 
    {
      return x + 1;
    }
  \end{lstlisting}
  \item Lo stesso nome può denotare oggetti differenti (es: tipi dinamici in Python)
  \item Annidamento $\rightarrow$ annido diversi blocchi di codice $\rightarrow$ dichiarazione visibile nel blocco di definizione e negli annidati
  \item \textbf{Suddivisione} dell'ambiente:
  \begin{itemize}
    \item ambiente locale $\rightarrow$ binding creati all'ingresso nel blocco $\rightarrow$ variabili \textbf{locali} e parametri \textbf{formali}
    \item ambiente globale $\rightarrow$ dichiarazioni esplicite di varlabili globali oppure binding ottenuti da moduli
  \end{itemize} 

  \newpage

  \item Operazioni sull'ambiente:
  \begin{itemize}
    \item \textbf{creazione} $\rightarrow$ associazione del nome ad un oggetto denotato (naming)
    \item \textbf{distruzione} $\rightarrow$ in uscita dal blocco (unnaming)
    \item \textbf{riferimento} $\rightarrow$ uso di un nome all'interno del codice per riferirsi ad un oggetto (referencing)
  \end{itemize}
  \item Tempo di vita dell'oggetto:
  \begin{itemize}
    \item può essere più lungo della vita del binding $\rightarrow$ oggetti chiamati prima e dopo l'esecuzione di una funzione
    \begin{lstlisting}
      int A;
      void P(int *X){
      ...
      }
      ...
      P(&A);
    \end{lstlisting}
    \item può essere più breve della vita del binding $\rightarrow$ puntatore a memoria deallocata dinamicamente
    \begin{lstlisting}
      int *X, *Y, z;
      X = (int *) malloc (sizeof (int));
      Y = X;
      *Y = 5;
      free (X);
      z = *Y;       
    \end{lstlisting}
  \end{itemize} 
  \item \textbf{Dangling reference} $\rightarrow$ riferimento pendente che può causare errori
\end{itemize}

\section*{Regole di Scope}
\begin{itemize}
  \item Scope \textbf{statico} $\rightarrow$ nome non locale determinato dai blocchi che lo racchiudono (a partire dall'interno)
  \begin{lstlisting}
    int x = 0;
    void foo (int n) {
    x = x + n;
    }
    foo(2);
    write(x);
    { int x = 0;
      foo (3);
      write (x);
    }
    write (x);
    //output del codice: "2" "0" "5"

  \end{lstlisting}
  \item Scope \textbf{dinamico} $\rightarrow$ nome non locale determinato dalla sequenza di blocchi attivi, a partire dal più recente
  \begin{lstlisting}
    int x = 0;
    void foo (int n) {
    x = x + n;
    }
    foo(2);
    write(x);
    { int x = 0;
      foo (3);
      write (x);
    }
    write (x);
    //output del codice: "2" "3" "2"
  \end{lstlisting}

  \newpage

\end{itemize}

\section*{Scope Statico vs Dinamico}
\begin{itemize}
  \item Scope \textbf{statico} (lexical scoping):
  \begin{itemize}
    \item informazioni complete ricavabili dal testo del programma
    \item binding effettuato a tempo di compilazione
    \item complesso da implementare ma efficiente $\rightarrow$ usato in praticamente tutti i linguaggi moderni (C, Java, \dots)
  \end{itemize}
  \item Scope \textbf{dinamico}:
  \begin{itemize}
    \item informazione ricavata dall'esecuzione
    \item flessibile $\rightarrow$ comportamento di una procedura facilmente modificabile
    \item semplice da implementare ma meno efficiente $\rightarrow$ usato in R o Ruby
  \end{itemize}
  \item Python $\rightarrow$ si può usare una variabile senza dichiararla in precedenza $\rightarrow$ l'assegnazione comporta la dichiarazione
  \begin{lstlisting}
    def f():
      write(x)
    def g():
      x = "local"
      f()
    x = "global"
    g()
    //stampa "global"
  \end{lstlisting}
\end{itemize}

\section*{Aliasing e Overloading}
\begin{itemize}
  \item Aliasing $\rightarrow$ nomi diversi denotano lo stesso oggetto (parametri o puntatori)
\end{itemize}

  \begin{multicols}{2}
    \begin{lstlisting}
      //Passaggio di parametri
      int x = 2;
      int foo (ref int y)
      {
        y = y + 1;
        x = x + y;
      }
      foo(x);
      write (x);
    \end{lstlisting}
    \columnbreak 
    \begin{lstlisting}
      //Uso di puntatori
      int *X, *Y;
      X = (int *) malloc (sizeof (int));
      *X = 0;
      Y = X;
      *Y = 1;
      write (*X);
    \end{lstlisting}
  \end{multicols}

\smallskip

\begin{itemize}
  \item Overloading $\rightarrow$ stesso nome usato con significati diversi a seconda del contesto
  \item il $+$ è un esempio di overloading (somma tra interi $\mid$ concatenazione di stringhe)
\end{itemize}

\section*{Let in Scheme}
\begin{itemize}
  \item Let $\rightarrow$ dichiarazione attraverso diversi costrutti
  \begin{itemize}
    \item let $\rightarrow$  non ricorsivo, crea in blocco il nuovo ambiente
    \item let$*$ $\rightarrow$  non ricorsivo, crea una serie di ambienti in modo sequenziale
    \item letrec $\rightarrow$ dichiarazione ricorsiva e mutuamente ricorsiva
  \end{itemize}
\end{itemize}

\begin{multicols}{2}
  \begin{lstlisting}
    //Output: 1
    (let ((a 1))
      (let ((a 2)
          (b a))
      b))
  \end{lstlisting}
  \columnbreak 
  \begin{lstlisting}
    //Output: 2
    (let ((a 1))
      (let* ((a 2)
        (b a))
      b))
  \end{lstlisting}
\end{multicols}

\newpage

\section*{Mutua ricorsione di tipi e funzioni}
\begin{itemize}
  \item Mutua ricorsione di funzioni:
  \begin{itemize}
    \item forza l'utilizzo di un nome \textbf{prima} che venga dichiarato
    \item possibile se si permettono \textbf{eccezioni al vincolo}
  \end{itemize}
  \begin{lstlisting}
    void f(){...g();} // g non ancora dichiarato
    void g(){...f();}
  \end{lstlisting}
  \item Mutua ricorsione di tipo:
  \begin{itemize}
    \item posso definire un puntatore prima di aver definito il tipo puntato
  \end{itemize}
  \begin{lstlisting}
    struct child {struct parent *pParent;}; //parent non ancora definito
    struct parent {struct child *children[2];};
  \end{lstlisting}
\end{itemize}

\section*{Dichiarazioni incomplete}
\begin{itemize}
  \item Dichiarazione incompleta di \textbf{tipo}:
  \begin{itemize}
    \item Dichiaro un tipo prima di averlo effettivamente definito:
  \end{itemize}
  \begin{lstlisting}
    typedef struct child ch; //ch definito dopo
    struct child {struct parent *pParent;};
    struct parent {ch *children[2];}
  \end{lstlisting}
  \item Dichiarazione incompleta di \textbf{funzione}:
  \begin{itemize}
    \item La prima dichiarazione della funzione è incompleta $\rightarrow$ serve per capire tipo dei parametri e valore restituito
  \end{itemize}
  \begin{lstlisting}
    void eval_parent(struct parent p); // solo dichiarazione
    
    void eval_child(ch c)
    {
      ...
      eval_parent(p2);
    }

    void eval_parent(struct parent p) // definizione completa
    { 
      ...
      eval_child(c2);
    }
  \end{lstlisting}
\end{itemize}

\end{document}
\documentclass{article}
\usepackage[top=1.5cm, bottom=1.5cm, left=1cm, right=1cm]{geometry}
\usepackage{graphicx}
\usepackage{listings}
\usepackage{color}


\definecolor{dkgreen}{rgb}{0,0.6,0}
\definecolor{gray}{rgb}{0.5,0.5,0.5}
\definecolor{mauve}{rgb}{0.58,0,0.82}

\lstset{
  frame=none,
  language=C,
  aboveskip=3mm,
  belowskip=3mm,
  showstringspaces=false,
  columns=flexible,
  basicstyle={\normalsize\ttfamily},
  numbers=none,
  numberstyle=\tiny\color{gray},
  keywordstyle=\color{blue},
  commentstyle=\color{dkgreen},
  stringstyle=\color{mauve},
  breaklines=true,
  breakatwhitespace=true,
  tabsize=3
}

\title{Astrazione sul controllo}
\author{R.S}

\begin{document}
\maketitle

\section*{Passaggio di parametri}

\begin{flushleft}
    
\textbf{Passaggio di parametri per valore:} \\
\begin{itemize}
    \item il parametro formale è una variabile locale, il parametro attuale viene valutato con r-value
    \item le modifiche al parametro formale non si ripercuotono su quello attuale
	\item anche dati grossi vengono copiati a scapito dell'efficienza (formale trattato come costante)
\end{itemize}

\medskip

\textbf{Passaggio di parametri per riferimento:} \\
\begin{itemize}
    \item il parametro è un puntatore ad un riferimento di memoria
    \item posso utilizzarlo con parole chiave (\&, ref..) oppure simulandolo coi puntatori
    \begin{lstlisting}
        void foo (ref int x){ x = x+1;}
        ...
        y = 1;
        foo(y);
        foo(V[y+1]);
    \end{lstlisting}
\end{itemize}

\medskip

\textbf{Passaggio di parametri per sharing:} \\
\begin{itemize}
    \item l'assegnazione crea la condivisione di una copia (Java, Python…), stessa cosa il passaggio per valore
    \item implementato ad esempio in array e liste
    \begin{lstlisting}
        def f(a_list):
            a_list += [1]
        m = []
        f(m)
        print(m)
    \end{lstlisting}
 
\end{itemize}

\medskip

\textbf{Call by name:} \\
\begin{itemize}
    \item l'espressione del parametro viene passata senza valutazione
    \item una chiamata esegue il corpo della procedura P valutando il parametro soltanto quando viene utilizzato (di default in Algol-W)
    \item la valutazione del parametro viene fatta ogni volta che il parametro viene utilizzato (effetti collaterali ripetuti)
\end{itemize}

\medskip

\textbf{Chiusura:} la coppia [exp, env] dove exp è il puntatore al codice dell'espressione ed env è un puntatore di catena statica a RdA \\

\medskip

\textbf{Call by  need:}
\begin{itemize}
    \item l'argomento viene valutato al massimo una volta
    \item le valutazioni multiple restituiscono sempre lo stesso risultato perché la prima valutazione viene conservata
    \item più efficiente del call by name, utilizzati nei linguaggi funzionali
\end{itemize}

\medskip

\end{flushleft}
 

\section*{Funzioni di ordine superiore}

\begin{flushleft}

Si possono passare funzioni come parametro ad altre funzioni e possono essere anche restituite come risultato

\medskip

\textbf{Semantica}
\begin{itemize}
    \item la funzione f viene passata come parametro e valutata eventualmente più volte
    \item problemi di scope delle variabili utilizzate da f (scope statico oppure dinamico)
\end{itemize}

\medskip

\textbf{Binding}
\begin{itemize}
    \item l'ambiente è quello della chiamata di g con parametro la funzione f $\rightarrow$ deep binding
    \item l'ambiente è quello della chiamata di f dentro g $\rightarrow$ shallow binding
\end{itemize}

\textbf{Implementazione della chiusura}
\begin{itemize}
    \item nel RdA di g si inserisce la chiusura [code, env] della funzione f
    \item alla chiamata di f si alloca il suo record di attivazione con puntatore alla catena statica env
    \begin{lstlisting}
        int x = 1;
        int f (int y){ return x + y; }
        int g (int h (int i)){
        int x = 2;
        return h(3) + x;}
        ...
        int x = 4;
        int z = g(f);
    \end{lstlisting}
\end{itemize}

\medskip

\textbf{Politiche di binding}

Ambiente di valutazione della funzione f (con nome h) dentro g:
\begin{itemize}
    \item scope statico $\rightarrow$ ambiente della definizione
    \item scope dinamico: \begin{itemize}
        \item al momento della creazione del legame h $\rightarrow$ f ovvero alla chiamata di f con parametro g
              $\rightarrow$ \textbf{deep binding}
        \item al momento della chiamata di f in g tramite h $\rightarrow$ \textbf{shallow binding}
    \end{itemize}
\end{itemize}

\medskip

\textbf{Implementazione dello scope dinamico}

L'implementazione può avvenire sia con deep che con shallow binding:

\begin{itemize}
    \item Deep binding: 
    \begin{itemize}
        \item alla funzione passata come parametro si associa la chiusura per mantenere in memoria l'RdA della chiamata
        \item per le procedure chiamate tramite parameteo si usa un link statico $\rightarrow$ salto della pila
    \end{itemize}
    \item Shallow binding: 
    \begin{itemize}
        \item non necessita la chiusura $\rightarrow$ si associa soltanto il codice della funzione
        \item per accedere a x si risale la pila $\rightarrow$ uso di A-List e CRT
    \end{itemize}
\end{itemize}

\medskip

\textbf{Gestione delle eccezioni}

I primi linguaggi utilizzavano l'istruzione \textbf{GOTO} $\rightarrow$ difficile impementare l'uscita da una procedura \\
Introduzione delle \textbf{eccezioni}:

\begin{itemize}
    \item definizione dei blocchi protetti $\rightarrow$ insiemi di eccezioni e relativi gestori
    \item il programma solleva un'eccezione $\rightarrow$ la computazione viene interrotta e si cerca un gestore
    \item gestore: 
    \begin{itemize}
        \item codice da eseguire soltanto in caso di eccezione relativa ad un blocco protetto
        \item ricerca del gestore $\rightarrow$ terminazione dei comandi, uscita da cicli e deallocazione di RdA
    \end{itemize}
\end{itemize}

Se l'eccezione non viene gestita nella procedura corrente $\rightarrow$ ri-sollevata nel punto di chiamata e propagata lungo la 
catena dinamica fino a quando non si raggiunge un gestore oppure il top level

\end{flushleft}

\end{document}







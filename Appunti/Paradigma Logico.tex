\documentclass[12pt]{extarticle}
\usepackage[top=1.5cm, bottom=1.5cm, left=1cm, right=1cm]{geometry}
\usepackage{graphicx}
\usepackage{listings}
\usepackage{color}
\usepackage{multicol}


\definecolor{dkgreen}{rgb}{0,0.6,0}
\definecolor{gray}{rgb}{0.5,0.5,0.5}
\definecolor{mauve}{rgb}{0.58,0,0.82}

\lstset{
  frame=none,
  language=C,
  aboveskip=3mm,
  belowskip=3mm,
  showstringspaces=false,
  columns=flexible,
  basicstyle={\normalsize\ttfamily},
  numbers=none,
  numberstyle=\tiny\color{gray},
  keywordstyle=\color{blue},
  commentstyle=\color{dkgreen},
  stringstyle=\color{mauve},
  breaklines=true,
  breakatwhitespace=true,
  tabsize=3
}

\title{Paradigma Logico}
\author{R.S}

\begin{document}
\maketitle

\section*{Sintassi}

\begin{itemize}
  \item Si usa la sintassi della logica del primo ordine $\rightarrow$ \textbf{predicate calculus}
  \item Componenti della sintassi:
  \begin{itemize}
    \item Simboli logici (uguali per tutti i linguaggi): 
    \begin{itemize}
      \item connettivi $\rightarrow$ $\wedge, \vee , \lnot, \dots$ 
      \item quantificatori $\rightarrow$ $\forall, \exists$
      \item costanti $\rightarrow$ true, false
      \item variabili $\rightarrow$ lettere maiuscole come X,Y,Z
    \end{itemize}
    \item Simboli non logici:
    \begin{itemize}
      \item predicati $\rightarrow$ padre(X,Y)
      \item funzioni $\rightarrow$ $f(a,b)$
      \item costanti $\rightarrow$ a, b, \dots
    \end{itemize}
    \item Termini $\rightarrow$ elementi base della sintassi: 
    \begin{itemize}
      \item una variabile può essere un termine, così come una costante
      \item i termini possono essere \textbf{composti} applicando funzioni ad altri termini $\rightarrow$ $f(a,X)$
      \item termini \textbf{ground} $\rightarrow$ termini che non contengono variabili (es. $f(a,b)$)
    \end{itemize}
    \item Formule:
    \begin{itemize}
      \item atomi $\rightarrow$ applicazione di un predicato $\rightarrow$ fratello(X,Y)
      \item formule complesse $\rightarrow$ fratello(X,Y) $\wedge$ fratello(Y,Z) 
      \item quantificazioni $\rightarrow$ $\forall$ X $\exists$ Y padre(X,Y)
    \end{itemize}
    \item  Clausole:
    \begin{itemize}
      \item definizione: $H:$ -$ A_1, A_2, \dots, A_n$ con $n \ge 0$ 
            \\ la definizione viene interpretata come $A_1 \wedge A_2 \wedge \dots \wedge A_n \rightarrow H$
      \item esempio di clausola definita:
      \begin{lstlisting}
        genitore(X,Y) :- padre(X,Y)
        antenato(X,Y) :- genitore(X,Y)
        antenato(X,Y) :- genitore(X,Z), antenato(Z,Y)
      \end{lstlisting}
      \item Fatti $\rightarrow$ clausole con corpo \textbf{vuoto}
      \item Programma Logico $\rightarrow$ insieme di clausole definite
      \item Goal (query) $\rightarrow$ sequenza di atomi da valutare e dimostrare
      \begin{itemize}
        \item esempio: ?- genitore(X,mario)
      \end{itemize}
    \end{itemize}
  \end{itemize}
\end{itemize}

\newpage

\section*{Variabili logiche e unificazione}
\begin{itemize}
  \item Applicare una clausola $\rightarrow$ unificare la query con la clausola stessa
  \item Variabile logica: può essere istanziata una sola volta $\rightarrow$ può essere associata a termini non ground
  \item Unificazione $\rightarrow$ soluzione di equazione tra termini
  \begin{itemize}
    \item equazione $\rightarrow$ $f(X,b) = f(a, Y)$, soluzione $\rightarrow$ $\{X/a, Y/b\}$
  \end{itemize}
  \item  Algoritmo di \textbf{Unificazione} (Martelli-Montanari)
  \begin{itemize}
    \item determina se un insieme di equazioni tra termini è risolvibile
    \item restituisce una \textbf{forma risolta} $\rightarrow$ MGU (most general unifier)
    \item obiettivo:
    \begin{itemize}
      \item insieme $\{s_1 = t_1, \dots, s_n = t_n\}$
      \item restituzione della MGU $\theta$ tale che $s_i\theta = t_i\theta$ $\forall i$ (fallimento altrimenti)
    \end{itemize}
    \item algoritmo incrementale $\rightarrow$ riduce la complessità dei termini ad ogni passo
    \item crea valori \textbf{parziali} che vengono poi raffinati
    \item è la base del motore di inferenza dei linguaggi logici
  \end{itemize}
  \item Sostituzione:
  \begin{itemize}
    \item una \textbf{sostituzione} è una funzione che associa variabili a termini
    \item definizione: $\theta = \{X_1/t_1, X_2/t_2, \dots X_n/t_n\}$
    \item esempio: $\theta = \{X/a, Y/f(b)\}$ $\rightarrow$ $\theta(X) = a, \theta(Y) = f(b)$
    \item applicazione di una sostituzione: si denota con $t\theta$ il termine che si ottiene sostituendo \\
          tutte le occorrenze di t secondo $\theta$
    \item \textbf{composizione} di sostituzioni:
    \begin{itemize}
      \item due sostituzioni $\rightarrow$ $\theta = \{X/a\}$ e $\sigma = \{Y/X\}$
      \item composizione $\rightarrow$ $\theta \circ \sigma = \{Y/a, X/a\}$
    \end{itemize}
  \end{itemize}
\end{itemize}

\section*{Risoluzione SLD}
\begin{itemize}
  \item parte da un goal iniziale e da un programma $\rightarrow$ risolve i goal scegliendo clausole appropriate
  \item una sequenza di applicazioni produce una \textbf{dimostrazione} (SLD refutation)
  \item la dimostrazione viene fatta per backtracking $\rightarrow$ ricerca di alternative in caso di fallimento
  \item esempio ProLog:
  \begin{lstlisting}
    //definizione
    antenato(X,Y) :- genitore(X,Y)
    antenato(X,Y) :- genitore(X,Z), antenato(Z,Y)
    //fatti definiti
    genitore(luca, mario)
    genitore(mario, giulia)
    //query
    ?- antenato(luca, giulia) //risposta attesa -> TRUE
  \end{lstlisting}
\end{itemize}

\newpage

\section*{Universo di Herbrand}
\begin{itemize}
  \item Insieme di tutti i termini costruibili da un insieme di simboli
  \item Termini costruiti a partire da costanti e funzioni $\rightarrow$ tutti ground
  \item esmepio: $\{a, f(a), f(f(a)), f(f(f(a))) \dots\}$
\end{itemize}

\section*{Interpretazione dichiarativa e procedurale}
\begin{itemize}
  \item Dichiarativa $\rightarrow$ significato \textbf{logico} del programma
  \item Procedurale $\rightarrow$ esecuzione vera e propria del programma
  \item Chiamate procedurali:
  \begin{itemize}
    \item chiamata procedurale $\rightarrow$ richiesta di \textbf{dimostrazione} di un goal
    \item se esiste una clausola compatibile continuo la computazione, altrimenti la chiamata fallisce
  \end{itemize}
  \item Computazione con risoluzione SLD
  \begin{itemize}
    \item regola inferenziale centrale per la computazione logica
    \item una derivazione \textbf{SLD} produce una sequenza di goal via via più semplici
    \item una \textbf{refutazione} SLD termina con un goal vuoto (successo)
    \item risultato $\rightarrow$ sostituzione dei valori ai parametri (answer substitution)
  \end{itemize}
  \item Backtracking:
  \begin{itemize}
    \item computazione logica \textbf{non deterministica}
    \item fallimento $\rightarrow$ torna indietro per esplorare alternative differenti
    \item inefficienza $\rightarrow$ dovuta alla quantità combinatoria di passi
  \end{itemize}
\end{itemize}

\section*{Predicato append in PROLOG}
\begin{itemize}
  \item Si vuole definire un predicato \textbf{append} tale che $append(L_1, L_2, Result)$ sia oddisfatto quando Result è la concatenazione
        di $L_1$ e $L_2$
  \item Definizione del predicato:
   \begin{itemize}
    \item caso base $\rightarrow$ concatenare una lista vuota a Ys
    \item caso ricorsivo $\rightarrow$ aggiungere X in testa e ricorsione su Xs
  \end{itemize}
  \begin{lstlisting}
    append([], Ys, Ys) . append([X|Xs], Ys, [X|Xs]) :- append(Xs, Ys, Zs)
  \end{lstlisting}
 \end{itemize}

 \newpage

 \section*{Non determinismo e aspetti di controllo}
\begin{itemize}
  \item I programmi logici sono \textbf{non deterministici} $\rightarrow$ non specificato ordine di applicazione delle 
        clausole
  \item Semantica logica indipendente dal controllo $\rightarrow$ strategie diverse portano a stessi risultati
  \item ProLog $\rightarrow$ non determinisimo gestito dall'interprete:
  \begin{itemize}
    \item clausole provate nell'ordine in cui appaiono nel codice
    \item atomi delle regole valutati da sinistra a destra
  \end{itemize}
  \begin{lstlisting}
    p(X) :- a(X).
    p(x) :- b(X), c(X).
    //valutato prima a(X), poi b(X)
    //se a(X) diverge allora tutto il programma diverge
  \end{lstlisting}
\end{itemize}

\section*{Operatore CUT (!)}
\begin{itemize}
  \item Elimina le alternative del backtracking $\rightarrow$ rende la computazione deterministica da quel momento
  \begin{lstlisting}
  //esempio di uso del CUT
  max(X, Y, X) :- X >= Y, !
  max(_, Y, Y).
  //se il >= risulta vero il cut non considera la seconda regola
  \end{lstlisting}
  \item Vantaggi:
  \begin{itemize}
    \item efficienza migliorata $\rightarrow$ meno operazioni di backtracking
    \item controllo esplicito e deterministico sul comportamento del programma
  \end{itemize}
  \item Svantaggi:
  \begin{itemize}
    \item la purezza logica è compromessa
    \item ragionamento dichiarativo complicato $\rightarrow$ male necessario
  \end{itemize}
\end{itemize}

\section*{Funzioni predefinite in PROLOG}
\begin{itemize}
  \item Sono le operazioni aritmetiche di base e l'operazione \textbf{is} e i predicati speciali (assert, write, read, retract)
  \item L'operatore is forza la valutazione di un'espressione:
  \begin{lstlisting}
    five(X) :- X is 2+3
    five(X) :- X = 2+3
  \end{lstlisting}
  \item Le funzioni predefinite sono semplici nella dichiarazione ma \textbf{inefficienti computazionalmente}
  \item Possibili ottimizzazioni:
  \begin{itemize}
    \item uso intelligente delle clausole e dei predicati ottimizzati
    \item riduzione dello spazio di ricerca utilizzando il \textbf{cut}
  \end{itemize}
\end{itemize}

\newpage

\section*{Programmazione con vincoli (CLP)}
\begin{itemize}
  \item Estensione della programmazione logica $\rightarrow$ maggiore efficienza per certi problemi
  \item Gestione \textbf{esplicita} di vincoli:
  \begin{itemize}
    \item vincoli \textbf{numerici} $\rightarrow$ CLP(R), CLP(Q)
    \item vincoli su insiemi finiti $\rightarrow$ CLP(FD)
    \item esempio:
    \begin{lstlisting}
      X + Y #= 10, X #> 0, Y #> 0
    \end{lstlisting}
  \end{itemize}
\end{itemize}

\section*{Vantaggi e Svantaggi del paradigma logico}
\begin{itemize}
  \item Vantaggi:
  \begin{itemize}
    \item programmi chiari, semplici e \textbf{dichiarativi}
    \item adatto a problemi di ricerca, IA e scheduling/ottimizzazione
    \item separazione tra logica e istruzioni di controllo
  \end{itemize}
  \item Svantaggi:
  \begin{itemize}
    \item potenziale inefficienza computazionale
    \item gestione del controllo complessa se senza esperienza
    \item difficoltà di \textbf{debugging} rispetto ai classici linguaggi imperativi
  \end{itemize}
  \item Conclusioni:
  \begin{itemize}
    \item approccio alternativo alla programmazione 
    \item utile per modellare problemi complessi in modo dichiarativo
    \item integra altri paradigmi nelle situazioni adatte
  \end{itemize}
\end{itemize}

\end{document}